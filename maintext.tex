\section{Introduction}
This paper is meant to be a give a very brief introduction to how Grasp can be seen in the context of other medical devices and technology applied today improve a patients quality of health. The emphasis is put on medical areas  where effort has been spent on testing and analyzing the the reported benefits of using Grasp. A few qualitative studies \cite{Guribye:2018:TID:3173225.3173287}, \cite{guribye2016}  with relatively small sample size has demonstrated how the Grasp can be used for logging and concurrent as well as retrospective reporting of affective events as a real time low threshold communication device for patients. Krøger \cite{kroger2015logging} performed a study with expert users (therapists and scholars) aimed at identifying how captured emotional events could be represented visually to enhance a diagnostic and therapeutic context.

\section{Mental well beeing - framing}
Below we will bring forth some experts from papers describing the use case, explanation of a transitional object and the experience sampling methods behind the design and use of Grasp. References contained in them can be found in those papers, as well as in this paper as they are re-referenced for convenience.

\subsection{Excerpt - use case }

"The use case that has been envisioned for the Grasp platform is a therapeutic context, i.e. a patient-therapist relationship, in which the platform can be used to alleviate recall bias. In this context, the Grasp platform can help the patient recall past events and emotional states, provide the therapist with a possible starting point for a conversation, and allow the two to collaboratively analyze changes and patterns in the collected data." \cite[p.~7]{guribye2016}

\subsection{Excerpt - transitional object }

"A person undergoing therapy who might benefit from using the Grasp to serve as a transitional object \cite{Arthern} that stores emotional events receives a Grasp at a clinical visit. The therapist instructs on use, and they jointly agree upon affective states or emotional events that should be registered until the next visit. During this period, the stone functions as an extended memory and as a transitional object." \cite[p.~7]{guribye2016}

\subsection{Excerpt - experience sampling }
In psychological and behavioral research, the umbrella term experience sampling methods (ESM) is used by Scollon, Prieto  and  Diener  \cite{scollon2003}  to  denote  a  number  of  self-reporting techniques.   These methods can be divided into three categories according to variations in the time of data registration:interval-contingent,  event-contingent,  and  signal-contingent sampling.  Respectively, data collection occurs according to a  set  time  interval,  when  a  specific  event  occurs  or  when prompted by a random signal \cite{scollon2003}.  The latter is also called ecological momentary assessment (EMA; see \cite{shiffman2008}).  An ad-vantage of this method is the possibility of capturing patterns that pertain to the recorded emotions, such as spatial, temporal, or situational correlates \cite{scollon2003}. \cite[p.~2]{guribye2016}

\subsection{Summary - framing of Grasp by third parties}
In a survey of technology for mental well being by Woodward et al \cite[p.~6]{DBLP:journals/corr/abs-1905-00288} they describe how participants "... were able to squeeze Grasp whenever they felt stressed and the device detected how much pressure was exhorted and displayed this data on a mobile app. This device allowed users to quickly and easily record their anxiety in real time which could be useful for monitoring stress over long periods as it does not rely on participants recording stressful events in a diary."

While developing technology aimed at reducing stress levels Rodrigues et al \cite[p.~1]{Rodrigues:2019:LTS:3341162.3343773} frame the Grasp as a device that can help identify stress.


\section{Other medical areas - a holistic perspective}
While we have focused mainly on the mental health use case and methodology in the paper so far we would also like to point out the need simpler ways of performing self reporting with regards to palliative care and other areas where self reporting of affective data is important for accurately determining correct diagnosis. As put forth by Espeland (MD) and Gjøsæter \cite{gjosater2018esrii} at the European Society for Research on Internet Interventions in 2018, we increasingly see remote patient monitoring with a strong emphasis on bio-metric data. Having tools that provide context to a continuous stream of data allowing the therapists to easily see the corrolation between emotional events and biometric data is very helpful.


\section{Ongoing efforts}
Currently we are designing an intermediate size clinical study investigating and measuring the perceived health benefits of Grasp. Research 

\section{Concluding remarks}
Through research performed in collaboration with Prof. Frode Guribye at the University of Bergen, Norway we have described what the Grasp technology is and how it can be employed for therapy in a general sense with the emphasis on its qualities with regards to as a low threshold pervasive affective "in situ active intentional" \cite[p.~2]{guribye2016}  sensing device designed to enable ecological momentary assessment as well as retroactive analysis and diagnostic use of this data.





\balance{}

